\documentclass[../exam_questions.tex]{subfiles}

\begin{document}
\lecture{Lecture 6}
\section{Define transitivity of a network in any of equivalent ways.}
Transitivity measures how strongly a network tends to form triangles - weather "a fried of my friend is also my friend".
\begin{definition}{Transitivity}{}
	Transitivity is a global measure defined using any of the following formulas
	\begin{align}
		C & = \frac{\text{number of closed paths of length 2}}{\text{number of paths of length 2}} \\
	\end{align}
	This formula shows a probability that two neighbors of the same node are connected. Among all two-step paths $A-B-C$, how many are closed by an edge $A-C$?
	\begin{align}
		C & = \frac{\text{number of triangles $\times$ 6}}{\text{number of paths of length 2}} \\
	\end{align}
	Each triangle can have its vertices ordered in 6 ways. It has the same meaning as the previous definition.
	\begin{align}
		C & = \frac{\text{number of triangles $\times$ 3}}{\text{number of connected triples}} \\
	\end{align}
	A single triangle can be written as 3 connected triples, one centered at each node. The formula is then: how many closed triples among all triples.
\end{definition}

\section{Define local clustering.}
\begin{align}
	C_i & = \frac{\text{number of pairs of neighbors of vertex} \: i \:  \text{that are connected}}{\text{number of pairs of neighbors of vertex}\:i}
\end{align}

\pagebreak
\section{Explain the concept of reciprocity and how it relates to loops of length two.}
\begin{definition}{Reciprocity}{}
	A global measure of the graph $r$
	\begin{align}
		r & = \frac{1}{m} \sum_{i,j} A_{ij}A_{ji}.
	\end{align}
	Equals to the number of 2-loops normalized to the number of edges.\\
	It gives a probability that if there is an edge $(i,j)$ there is also an edge in the opposite direction $(j,i)$.
\end{definition}

\section{Define structural balance for networks with signed edges. Show that structurally balanced network is certainly clusterable (Harrary's theorem).}
Signed edges carry a positive or negative sign.\\
A graph is structurally balanced if all loops in the graph have even number of negative edges.\\
A signed network is clusterable if it is possible to clearly partition it into two subsets of vertices such that all edges between vertices within each subset are positive while all the edges between the two subsets are negative.\\
Clustering algorithm - color the vertices along paths using 2 colors, change the color when traversing a negative edge.
After finishing the loop, the color changed even times - so it has changed to the original color of the first vertex in the loop.

\section{Show by counterexample that a clusterable network need not be structurally balanced.}
A triangle with every edge negative. Each vertex is a cluster of 1 node. There are 3 clusters.
\begin{notebox}
	Definition above defines a 2-clusterable network, but during the lecture, the counterexample was this one exactly, with three clusters
\end{notebox}


\section{Explain vertex similarity.}
A measure defined between two vertices. There are two types of similarity
\begin{itemize}
	\item Structural equivalence
	\item Regular equivalence
\end{itemize}

\section{Define structural and regular equivalence and explain the differences of them.}
\subsection{Structural equivalence}
The amount of shared neighbors
\begin{align}
	n_{ij} & = \sum_k A_{ik}A_{kj} = (A^2)_{ij}.
\end{align}
Intended use for undirected graphs.\\
Gives the number of length 2 paths between the ordered pair of nodes.

\subsection{Regular equivalence}
Outcome of an iterative process
\begin{align}
	\sigma_{ij} & = \alpha \sum_{kl} A_{ik}A_{jl}\sigma_{kl} = \alpha \sum_{kl}A_{ik}\sigma_{kl}A_{lj}^T \\
	\sigma      & = \alpha A \sigma A^T
\end{align}

\section{Define homophily and assortative mixing in networks.}
\subsection{Homophily}
A global property based on similarity
\begin{itemize}
	\item Enumerative characteristics
	\item Scalar characteristics
\end{itemize}
It shows how likely are similar nodes to form ties.
\subsubsection{Enumerative characteristics}
Defines a finite number of distinct classes $c_i$.

We count the edges between the nodes of the same class
\begin{align}
	\sum_{i,j} \delta(c_i, c_j) = \sum_{i,j} A_{ij}\delta(x_i, c_j)
\end{align}
This number is compared to the number expected if connections were made at random
\begin{align}
	\frac{1}{2} \sum_{i,j} \delta(c_i, c_j).
\end{align}
The difference is
\begin{align}
	\frac{1}{2}\sum_{i,j} A_{ij}\delta(x_i, c_j) - \frac{1}{2} \sum_{i,j} \frac{d_i d_j}{2m} \delta (c_i, c_j).
\end{align}
The modularity is a global property defined as
\begin{align}
	Q & = \frac{1}{2m} \sum_{i,j} \left(A_{ij} - \frac{d_i d_j}{2m}\right) \delta(c_i, c_j).
\end{align}
A new algebraic object nonsparse modularity matrix $B$ describes network structure
\begin{align}
	B_{i,j} & = A_{i,j} - \frac{d_i d_j}{2m}.
\end{align}
The maximal value of modularity equals
\begin{align}
	Q_{\rm max} & = \frac{1}{2} \left(2m - \sum_{i,j} \frac{d_i d_j}{2m} \delta(c_i, c_j)\right)
\end{align}

\subsubsection{Scalar characteristics}
There are no distinct classes but rather a continuum of different values.
Scalar characteristics relates to covariance of characteristics (value) found on both ends of edges in the graph. It measures how correlated are the values of the vertices at the ends of a single edge.
For each edge $(i,j) \in E$ there are scalar values $x_i, x_j$ assigned to vertices $i, j$.
The mean value is
\begin{align}
	\mu & = \frac{\sum_{i,j} A_{ij} x_i}{\sum_{i,j} A_{ij}} = \frac{\sum_i d_i x_i}{\sum_i d_i} = \frac{1}{2m} \sum_i d_i x_i
\end{align}

The covariance is defined as
\begin{align}
	{\rm cov}(x_i, x_j) & = \frac{\sum_{i,j}A_{ij}(x_i - \mu)(x_j - \mu)}{\sum_{i,j}A_{ij}}        \\
	{\rm cov}(x_i, x_j) & = \frac{1}{2m} \sum{i,j}\left(A_{ij} - \frac{d_i d_j}{2m}\right)x_i x_j.
\end{align}

\subsection{Assortative mixing}
For example: based on node in-degree.
Shows how similar nodes (in this case nodes with the same in-degree) are connected together.

Assortatively mixed network will have a high in-degree nodes connected with each other. The remained of nodes will be poorly connected to each other and usually even to the core by some path.

Dissortatively mixed network will have high in-degree nodes (hubs) connected to vast amount of low in-degree nodes.

\end{document}