\documentclass[../exam_questions.tex]{subfiles}

\begin{document}
% \section{Define transitivity of a network in any of equivalent ways.}
\section{Define local clustering.}
\begin{align}
	C_i & = \frac{\text{number of pairs of neighbors of vertex} \: i \:  \text{that are connected}}{\text{number of pairs of neighbors of vertex}\:i}
\end{align}
\section{Explain the concept of reciprocity and how it relates to loops of length two.}
\begin{definition}{Reciprocity}{}
	A global measure of the graph $r$
	\begin{align}
		r & = \frac{1}{m} \sum_{i,j} A_{ij}A_{ji}.
	\end{align}
	Equals to the number of 2-loops normalized to the number of edges.
\end{definition}
\section{Define structural balance for networks with signed edges. Show that structurally balanced network is certainly clusterable (Harrary's theorem).}
Signed edges carry a positive or negative sign.\\
A graph is structurally balanced if all loops in the graph have even number of negative edges.\\
A signed network is clusterable if it is possible to clearly partition it into two subsets of vertices such taht all edges betwwen certices within each subset are positive while all the edges between the two subsets are negative.\\
Clustering algorithm - color the vertices along paths using 2 colors, change the color when traversing a negative edge.
After finishing the loop, the color changed even times - so it has changed to the original color of the first vertex in the loop.
\section{Show by counterexample that a clusterable network need not be structurally balanced.}

\section{Explain vertex similarity.}
A measure defined between two vertices
\begin{itemize}
	\item Structural equivalence
	\item Regular equivalence
\end{itemize}

\section{Define structural and regular equivalence and explain the differences of them.}
\subsection{Structural equivalence}
The amount of shared neighbors
\begin{align}
	n_{ij} & = \sum_k A_{ik}A_{kj} = (A^2)_{ij}.
\end{align}
Intended use for undirected graphs.\\
Gives the number of length 2 paths between the ordered pair of nodes.

\subsection{Regular equivalence}
Outcome of an iterative process
\begin{align}
	\sigma_{ij} & = \alpha \sum_{kl} A_{ik}A_{jl}\sigma_{kl} = \alpha \sum_{kl}A_{ik}\sigma_{kl}A_{lj}^T \\
	\sigma      & = \alpha A \sigma A^T
\end{align}

\section{Define homophily and assortative mixing in networks.}
\subsection{Homophily}
A global property based on similarity\\
\begin{itemize}
	\item Enumerative characteristics
	\item Scalar characteristics
\end{itemize}
\subsubsection{Enumerative characteristics}
The amount of edges between the nodes of the same class
\begin{align}
	\sum_{i,j} \delta(c_i, c_j) = \sum_{i,j} A_{ij}\delta(x_i, c_j)
\end{align}
This number is compared to the number expected if connections were made at random
\begin{align}
	\frac{1}{2} \sum_{i,j} \delta(c_i, c_j) = \sum_{i,j} A_{ij}\delta(x_i, c_j) -
	\frac{1}{2} \sum_{i,j} \frac{d_i d_j}{2m} \delta (c_i, c_j).
\end{align}
The modularity is a global property defined as
\begin{align}
	Q & = \frac{1}{2m} \sum_{i,j} \left(A_{ij} - \frac{d_i d_j}{2m}\right) \delta(c_i, c_j).
\end{align}
A new algebraic object nonsparse modularity matrix $B$ describes network structure
\begin{align}
	B_{i,j} & = A_{i,j} - \frac{d_i d_j}{2m.}
\end{align}
The maximal value of modularity equals
\begin{align}
	Q_{\rm max} & = \frac{1}{2} \left(2m - \sum_{i,j} \frac{d_i d_j}{2m} \delta(c_i, c_j)\right)
\end{align}

\subsection{Scalar characteristics}
\begin{align}
	\mu & = \frac{\sum_{i,j} A_{ij} x_i}{\sum_{i,j} A_{ij}} = \frac{\sum_i d_i x_i}{\sum_i d_i} = \frac{1}{2m} \sum_i d_i x_i
\end{align}
\end{document}