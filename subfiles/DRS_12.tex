\documentclass[../exam_questions.tex]{subfiles}

\begin{document}

\lecture{Lecture 12} % Video lecture 10
\section{Explain the difference between site and bond percolation.}
Percolations are stochastic processes. We distinguish between site and bond percolations which correspond to random removal of vertices or edges from a graph.
\begin{itemize}
	\item Site percolation - vertices together with all their pertaining edges, are removed from the graph randomly with probability $p$
	\item Bond percolation - edges are removed from the network randomly with probability $p$
\end{itemize}
We remove vertices or edges from the graph with increasing probability and observe how is the network affected.
The expected amount of removed vertices/edges is $np/mp$.
\begin{itemize}
	\item Phase transition - network transitions abruptly from being connected to being a set of disconnected components. The transition happens for some probability $p$
\end{itemize}
Percolations are used for modelling the spread of forest fires, spread of epidemics in a network.


\section{Describe uniform and non-uniform vertex removal. Comment on the robustness of power law networks to vertex removal.}
A modification of the usual model removes the vertices non-uniformly, for example by their degrees - $p_d$ is a probability that a vertex of degree $d$ is removed.

Power law configuration model is found to be remarkably resilient to random uniform removal of vertices but far more susceptible to specific targeted attacks. Removal of a small fraction of
highest-degree vertices can be extremely detrimental - attacking a hub.

\section{Describe a few examples of fully mixed epidemics models, e.g. SI, SIR, SIS, SEIR.}
Used for purposes of mathematical description of epidemics.

\subsection{SI}
\begin{itemize}
	\item Divides the population into two groups susceptible of size $S$ and infected of size $X$. The fraction of the total population is denoted by $s$ and $x$.
	\item The size of these groups cannot be negative
	\item No entry or departure of individuals from the population is assumed
	\item The system is compartmental - $x \in \R^n_{+},\: \dot{x} = f(x), \: x_i = 0 \Rightarrow f_i(x_i) \geq 0$ - (in this context $x$ is a general vector, not the fraction of infected)
\end{itemize}
The system is modelled as
\begin{align}
	\dot{s} & = -\beta s x \\
	\dot{x} & = \beta s x.
\end{align}
The coefficient $\beta$ represents the probability of a single infected infecting a susceptible individual. The probability of an infected person meeting a susceptible person is given by the sizes of each of the groups.
With constraint $s+x = 1$ the dynamics are reduced to a one dimensional system
\begin{align}
	\dot{x} & = \beta (1-x) x,
\end{align}
which has a closed form solution
\begin{align}
	x(t) & = \frac{x(0)\exp(\beta t)}{1 - x(0) + x(0) \exp(\beta t)}.
\end{align}

\subsection{SIR}
This model adds a group of recovered subpopulation of size $R$. The model is
\begin{align}
	\dot{s} & = -\beta s x           \\
	\dot{x} & = \beta s x - \gamma x \\
	\dot{r} & = \gamma x.
\end{align}
The model does not have a closed form solution.\\

\subsection{SIS}
The SIS model allows for recovered individuals to be infected again and so it is modelled as
\begin{align}
	\dot{s} & = \gamma x - \beta s x  \\
	\dot{x} & = \beta s x - \gamma x.
\end{align}
The constraint $s+x = 1$ implies
\begin{align}
	\dot{x} & = \beta (1-x) x - \gamma x       \\
	\dot{x} & = (\beta - \gamma) x - \beta x^2
\end{align}
which has a closed form solution
\begin{align}
	x(t) & = \left(1 - \frac{\gamma}{\beta}\right) \frac{c \exp[(\beta - \gamma) t]}{1+c \exp[(\beta - \gamma) t]}, \: {\rm where} \\
	c    & = \frac{\beta x_0}{\beta - \gamma - \beta x_0}
\end{align}
The system has a globally asymptotically stable equilibrium $x = 0$ for $\beta - \gamma < 0$.

\section{Explain how to model epidemics on networks. What is the role of vertices and vertex variables?}
Network epidemics models divide the whole population into subgroups having different contact patterns to give a detailed predictions including the effects of various management strategies that treat various segments of the population differently.
\begin{itemize}
	\item Each graph node describes a subpopulation $s_i, x_i, r_i, \dots$ - for example cities, regions or different groups in the population - seniors, children
	\item The spread across the network depends on the connectivity of the graph
	\item In networked epidemics models we use degree approximation - all nodes of the same degree are assumed to behave similarly regardless of their precise position in the network
\end{itemize}
\subsection{nSI}
Diagonal adjacency matrix elements $A_{ii} = 1$
\begin{align}
	\dot{s}_i & = - \beta s_i \sum_{k} A_{ij} x \\
	\dot{x}_i & = \beta s_i \sum_{j} A_{ij} x_j
\end{align}
\subsection{A stochastic variant of the networked model}
This variant uses correlations instead of products
\begin{align}
	\odv{}{t} <s_i> & = - \beta \sum_{j} A_{ij} <s_i x_j>
\end{align}

\section{Describe the Lotka-Volterra predator-prey model. Explain various ways how it extends to the network setting.}
A population dynamics model is given in a form
\begin{align}
	\dot{x}_i & = x_i \alpha_i(x_i, x_1, \dots, x_n),
\end{align}
where $\alpha_i$ is the growth rate for species $i$, depending generally on all other species' populations in the ecosystem. The whole system is a non-negative compartmental system.

The Lotka-Volterra model is defined as
\begin{align}
	\dot{x} & = x(\alpha - \beta y)       \\
	\dot{y} & = y ( - \gamma + \delta x),
\end{align}
where $x$ describes the prey population, $y$ describes the predator population and $\alpha, \beta, \gamma, \delta > 0$.\\
There exists a preserved quantity
\begin{align}
	V = -\delta x + y \ln x - \beta y + \alpha \ln y.
\end{align}
There is a single equilibrium state surrounded by a continuum of periodic orbits.
\subsection{Network variants}
\subsubsection{Multiple species}
\begin{align}
	\dot{x}_i & = x_i \left(\alpha_i + \sum_j A_{ij} x_j\right),
\end{align}
where $A_ij$ has signed edges
\begin{align}
	A_{ij}
	\begin{cases}
		> 0, \: \text{if species $i$ preys on species $j$} \\
		< 0, \: \text{if species $j$ preys on species $i$}
		% = 0, \: \text{if species $i$ and $j$ do not interact}
	\end{cases}.
\end{align}
Graph edges model predator-prey relations within one ecosystem.

\subsubsection{Patchy environment}
This variant models interaction of two species in multiple locations
\begin{align}
	\dot{x}_i & = x_i (\alpha - \beta y_i) + \sum_j A_{ij} x_j   \\
	\dot{y}_i & = y_i (-\gamma + \beta x_i) + \sum_j A_{ij} y_j,
\end{align}
where $x_i$ describes the size of prey population in location $i$. Graph edges model the migration from neighboring areas in proportion to populations in the $i^{\rm th}$ location.



\end{document}