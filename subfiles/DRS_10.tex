\documentclass[../exam_questions.tex]{subfiles}

\begin{document}

\chapter{Lecture 10}
\section{Describe the power method for finding leading eigenvalues and eigenvectors. What is the importance of choosing the initial seed?}
\TODO{Missing}

\section{Explain how to efficiently find all eigenvalues and eigenvectors of a given matrix. Specify which algorithms are used for matrix transformation and efficient solution of the transformed eigen-problem, given different starting matrices (symmetric/asymmetric, sparse/dense).}
\TODO{Missing}

\section{Why to use heuristic algorithms in general, even if no proof of correctness is available?}
Heuristic algorithms return a result which is not guaranteed to be best fo all cases, but for most practical purposes it is often good enough. It returns a fairly good divisions - approximate but acceptable solutions - no proof of validity is available\\
We use heuristic algorithms, when it comes to computationally difficult problems  - either the algorithm runs fast but fails to find the best solution almost always, or it always finds the best solution but takes prohibitively long time to return the result\\
Characteristically for heuristic algorithms, this assertion is not rigorously proven, hence it remains a conjecture, albeit one that points directly to the presumed fundamental difference between P and NP type problems

\section{What is the difference between the graph partitioning and the community detection problems?}
\subsection{Graph partitioning}
In graph partitioning, we are dividing graph vertices into a given number of non-overlapping groups of a fixed size. Number of inter-group edges is minimized.
Motivated by task allocation in distributed computing. Applications in network process simulations on parallel computers
\subsection{Community detection}
Number and sizes of groups are not given but are result of an algorithm. It is primarily used as a tool for analysis and understanding network data. The criterion of division can be defined in various ways, through the extent of modularity.
Reveals hidden structures in a network.

\section{Describe the Kernighan-Lin algorithm for graph partitioning. What is its computational complexity? What is roughly the size of the network for which it can be reasonably expected to work?}
\begin{enumerate}
	\item Divide vertices into 2 groups $V_1$ and $V_2$ of required sizes $n_1,\:n_2$ in any way
	\item For all pairs $i,j \in V_1, V_2$ calculate the change in cut set size between the groups if the vertices are interchanged
	\item From all such pairs find the one $(i,j)*$ that reduces the cut set size the most, or in absence of any such, the one that increases the cut size the least
	\item Swap the pair - this preserves assigned sizes of both groups
	\item Repeat the process from step 2 with the exception that the moved pairs cannot be moved again in this round
	\item Stop when there are no more pairs to swap
	\item When all swaps are completed - select from all partitions the one with the smallest cut set
	\item Repeat the whole process with this partition. Stop when there is no improvement
\end{enumerate}
\subsection{Complexity}
\begin{itemize}
	\item Number of swaps in each round is $\min(n_1, n_2) \in [0, \frac{n}{2}]$ resulting in $O(n)$ in the worst case
	\item For each swap the amount of pairs is $\frac{n^2}{2}$ in the worst case resulting in $O(n^2)$
	\item For each examined swap the reduction in cut set size is calculated $O(\frac{m}{n})$
	\item The total complexity for one round is $O(mn^2)$ which is $O(n^3)$ on sparse networks and $O(n^4)$ on dense networks
	\item It is applicable for $n\leq 10^3$
\end{itemize}

\section{Describe the spectral partitioning algorithm. Explain the importance of the graph Laplacian matrix and its Fiedler eigenvalue. What is its computational complexity? What is roughly the size of a network for which it can be reasonably expected to work?}
\begin{warningbox}
	Proof/Derivation is required as a part of the exam
\end{warningbox}
\begin{itemize}
	\item Spectral graph partitioning is a method for graph bipartition
	\item It is based on the graph Laplacian
	\item Assumes an undirected graph
	\item The sizes of the 2 clusters are given
	\item The cut set size equals
	      \begin{align}
		      R & = \frac{1}{2} \sum_{i,j} A_{ij}
	      \end{align}
\end{itemize}
\begin{enumerate}
	\item Define a vector $s \in \R^n$ with components $s_i$
	      \begin{align}
		      s_i & =
		      \begin{cases}
			      +1 \:\text{if}\: i \in V_1 \\
			      -1 \:\text{if}\: i \in V_2 \\
		      \end{cases}
	      \end{align}
	\item With the help of $s_i$ vertex labels one can construct the expression
	      \begin{align}
		      \frac{1}{2}(1 - s_i s_j) & =
		      \begin{cases}
			      1 \:\text{if} \:(i,j) \:\text{vertices are in different groups} \\
			      0 \:\text{if} \:(i,j) \:\text{vertices are in the same group}
		      \end{cases}
	      \end{align}
	\item We can then express the size of the cut set as
	      \begin{align}
		      R                      & = \frac{1}{2} \sum_{i,j = 1}^{n} A_{ij}\frac{1}{2}(1-s_i s_j) = \frac{1}{4} \sum_{i,j=1}{n} A_{ij}(1-s_i s_j) \\
		      \sum_{i,j=1}^{n}A_{ij} & = \sum_{i=1}^{n}d_i = \sum_{i=1}^{n}d_i s_i^2 = \sum_{i=1}^{n}d_i \delta_{ij} s_i s_j                         \\
		      R                      & = \frac{1}{4} \sum_{i=1}^{n} (d_i \delta_{ij} - A_{ij}) s_i s_j = \frac{1}{4} \sum_{i=1}^{n} L_{ij} s_i s_j   \\
		      R                      & = \frac{1}{4} s^T L s
	      \end{align}
\end{enumerate}
\subsection{Spectral Partitioning Algorithm}
\begin{enumerate}
	\item Calculate the Fiedler eigenvector $v \in \R^n$ of $L$
	\item Sort the elements $v_2i$ from largest to smallest
	\item Assign $n_1$ most positive elements to $V_1$ and the remaining elements to $V_2$
	\item Assign $n_1$ most negative elements to $V_1$ and the remaining elements to $V_2$
	\item Identify which partition results in smaller cut set size
\end{enumerate}
\subsection{Complexity}
\begin{itemize}
	\item Finding the Fiedler eigenvector can be done with $O(mn)$
	\item It is applicable for $n\leq 10^5$
	\item The smaller is the Fielder eigenvalue, the easier it is to partition a network
\end{itemize}
\end{document}