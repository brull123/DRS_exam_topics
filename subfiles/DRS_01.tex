\documentclass[../exam_questions.tex]{subfiles}

\begin{document}

\chapter{Lecture 1}
\section{Recount briefly the history of man-made networks.}
\TODO{Missing}

\section{Describe the general structure of a network.}
A network consists of nodes connected by nodes. We usually represent networks using graphs.

\begin{definition}{Graph}{}
	A graph $G(V,E)$ is a topological object, where $V$ is a finite set of vertices and $E$ is a set of graph edges.
	Vertices $i,j$ are called adjacent if those are connected by an edge $(i,j)$.
\end{definition}

\section{Name a few examples of physical, biological, social and information networks.}
\subsection{Physical}
\subsubsection{Power grids}
\begin{itemize}
	\item High-voltage lines
	\item Generating stations
	\item Switching substations
	\item Consumers
\end{itemize}

\subsubsection{Transportation networks}
\begin{itemize}
	\item Roads, railroads, airline routes, shipping lanes
	\item Vertices represent geographical locations - road intersections, etc.
	\item Edges represent an existing link available for the considered type of transport - road, railway
	\item Differences between road networks and airline networks - Achieving connectedness with shortest paths of limited length while keeping the total number of edges as low as possible - relative weight give to those two opposing goals determines the topology of the network - emergence of hubs in airline network as opposed to road infrastructure
\end{itemize}

\subsection{Biological}
\subsubsection{Neural network}
\begin{itemize}
	\item Vertices represent neurons
	\item Edges represent dendrites and axon synaptic interconnections
\end{itemize}
\subsubsection{Food webs}
\begin{itemize}
	\item Vertices represent species and edges represent predator-prey relations in an ecosystem
\end{itemize}
\subsubsection{Population networks}
\subsubsection{Epidemiological network models}

\subsection{Social}
Social networks are comprised of people, groups, classes, companies, entities and their relations. Those exhibit opinion formation and propagation dynamics
\begin{itemize}
	\item Affiliation networks - bipartite graph (groups and individuals) edges signify which individuals belong to which group
\end{itemize}

\subsection{Information}
\subsubsection{Telecommunication networks}
\begin{itemize}
	\item A group of nodes interconnected by telecommunication links that are used to exchange messages between the nodes.
	\item Telecom links can be point-to-point, broadcast or multipoint using circuit switching, message switching or packet switching
	\item Today, the most important telecom network is the Internet
\end{itemize}


\section{Describe how the structure of various networks is revealed empirically.}
\TODO{This section is mainly generated by AI - not sure what the intended answer is}
Analysis approach simplifies the networks to a pure structure - only the connection patterns are considered, properties which stem from connection patterns alone.

Once the data are gathered, the resulting network is analyzed using mathematical and statistical tools to reveal properties such as degree distributions, clustering, community structure, and path lengths. These empirical studies make it possible to understand both the architecture and behavior of real-world networks.


\end{document}