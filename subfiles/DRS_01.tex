\documentclass[../exam_questions.tex]{subfiles}

\begin{document}

\lecture{Lecture 1}
\section{Recount briefly the history of man-made networks.}
In the $19^{\rm th}$ there was no theory for networks as there was no need for it. The development of technologies such as the power grid, telegraph, etc.,
was in an early stage, where the focus was on function of a single object rather than the topology of a whole network.

In the $20^{th}$ century, the increase of motor and railroad vehicles lead to congestions. The need of finding
places with increased traffic lead to the need of graph analysis tools.

The Internet relies on the topological structure of networks - small world property (top-heavy distribution).

\section{Describe the general structure of a network.}
A network consists of nodes connected by nodes. We usually represent networks using graphs.

\begin{definition}{Graph}{}
	A graph $G(V,E)$ is a topological object, where $V$ is a finite set of vertices and $E$ is a set of graph edges.
	Vertices $i,j$ are called adjacent if those are connected by an edge $(i,j)$.
\end{definition}

\section{Name a few examples of physical, biological, social and information networks.}
\subsection{Physical}
\subsubsection{Power grids}
\begin{itemize}
	\item High-voltage lines
	\item Generating stations
	\item Switching substations
	\item Consumers
\end{itemize}

\subsubsection{Transportation networks}
\begin{itemize}
	\item Roads, railroads, airline routes, shipping lanes
	\item Vertices represent geographical locations - road intersections, etc.
	\item Edges represent an existing link available for the considered type of transport - road, railway
	\item Differences between road networks and airline networks - Achieving connectedness with shortest paths of limited length while keeping the total number of edges as low as possible - relative weight give to those two opposing goals determines the topology of the network - emergence of hubs in airline network as opposed to road infrastructure
\end{itemize}

\subsection{Biological}
\subsubsection{Neural network}
\begin{itemize}
	\item Vertices represent neurons
	\item Edges represent dendrites and axon synaptic interconnections
\end{itemize}
\subsubsection{Food webs}
\begin{itemize}
	\item Vertices represent species and edges represent predator-prey relations in an ecosystem
\end{itemize}
\subsubsection{Population networks}
\subsubsection{Epidemiological network models}

\subsection{Social}
Social networks are comprised of people, groups, classes, companies, entities and their relations. Those exhibit opinion formation and propagation dynamics
\begin{itemize}
	\item Affiliation networks - bipartite graph (groups and individuals) edges signify which individuals belong to which group
\end{itemize}

\subsection{Information}
\subsubsection{Telecommunication networks}
\begin{itemize}
	\item A group of nodes interconnected by telecommunication links that are used to exchange messages between the nodes.
	\item Telecom links can be point-to-point, broadcast or multipoint using circuit switching, message switching or packet switching
	\item Today, the most important telecom network is the Internet
\end{itemize}


\section{Describe how the structure of various networks is revealed empirically.}
The empirical approach consists of 2 stages: observation and analysis.

The observations are done using web crawlers - a breadth first search which discovers different websites referred to from other websites using hyperlinks.

The raw topological data are analyzed to reveal properties, patterns - centralities, degree distributions, clustering, assortativity

\end{document}