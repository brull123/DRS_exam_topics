\documentclass[../exam_questions.tex]{subfiles}

\begin{document}

\section{Define assortative mixing with respect to scalar characteristics. Explain how does it differ from that for enumerative characteristics.}
Assortatively mixed networks will have high in-degree nodes well connected to each other-comprising the network's core. The remaining
low in-degree nodes are poorly connected to each other but usually connected to the core via some path.

Dissortatively mixed networks usually consist of high-degree hubs weakly connected to each other but each connecting into many low in-degree nodes.
TODO

\subsection{Scalar characteristics}
The assortative coefficient is defined as
\begin{align}
	r & = \frac{\sum_{i,j} \left(A_{ij} - \frac{d_i d_j}{2m}\right)x_i x_j}{\sum_{i,j} \left(d_i \delta_{ij} - \frac{d_i d_j}{2m}\right)x_i x_j}
\end{align}

\section{Define the modularity matrix for undirected networks.}
\begin{align}
	B_{i,j} & = A_{i,j} - \frac{d_i d_j}{2m.}
\end{align}

\section{Explain assortative mixing with respect to the degree. Discuss what are its implications on the network topology.}
Assortatively mixed networks will have high in-degree nodes well connected to each other-comprising the network's core. The remaining
low in-degree nodes are poorly connected to each other but usually connected to the core via some path.

Dissortatively mixed networks usually consist of high-degree hubs weakly connected to each other but each connecting into many low in-degree nodes.


\section{Define the power law distributions, explain how to calculate their moments.}
\section{Explain the effect of top heavy distribution of vertex degrees for the networks.}
A minor proportion of highly connected vertices accounts for a sizeable portion of the graph edges.\\
Such networks are remarkably robust to random node failure, but extremely susceptible to targeted attacks.

\section{Explain the importance of the small world effect for the functionality of computer networks.}
\begin{definition}{Small world networks}{}
	\begin{itemize}
		\item It is an example of a regular graph - each vertex connects to an even number $c$ of symmetrically placed immediate neighbors
		\item The diameter scales as $\log n$
	\end{itemize}

\end{definition}
\TODO{Not complete}
\end{document}