\documentclass[../exam_questions.tex]{subfiles}

\begin{document}

\lecture{Lecture 7}
\section{Define assortative mixing with respect to scalar characteristics. Explain how does it differ from that for enumerative characteristics.}
\subsection{Scalar characteristics}
The assortative coefficient for scalar characteristics is defined as
\begin{align}
	r & = \frac{\sum_{i,j} \left(A_{ij} - \frac{d_i d_j}{2m}\right)x_i x_j}{\sum_{i,j} \left(d_i \delta_{ij} - \frac{d_i d_j}{2m}\right)x_i x_j}
\end{align}

\subsection{Enumerative characteristics}
The assortative coefficient for enumerative characteristics is defined as
\begin{align}
	\frac{Q}{Q_{\rm max}} & = \frac{\sum\left(A_{ij} - \frac{d_i d_j}{2m}\right) \delta (c_i, c_j)}{2m - \sum\frac{d_i d_j}{2m}\delta(c_i, c_j)}.
\end{align}

The difference is that enumerative characteristics allows for a finite set of distinct classes, while
scalar characteristics allows for a continuum of values and calculates the correlation between the values at the ends of edges rather then whether the classes are the same.

\section{Define the modularity matrix for undirected networks.}
\begin{align}
	B_{i,j} & = A_{i,j} - \frac{d_i d_j}{2m.}
\end{align}
It assumes an undirected graph. The matrix is always nonsparse regardless of the graph topology and it is used to calculate both scalar and enumerative characteristics.

\section{Explain assortative mixing with respect to the degree. Discuss what are its implications on the network topology.}
Assortatively mixed networks will have high in-degree nodes well connected to each other-comprising the network's core. The remaining
low in-degree nodes are poorly connected to each other but usually connected to the core via some path.

Dissortatively mixed networks usually consist of high-degree hubs weakly connected to each other but each connecting into many low in-degree nodes.


\section{Define the power law distributions, explain how to calculate their moments.}
A statistical characterization of a large network uses power-law degree distributions.

Probability $p_k$ of finding a vertex of degree $d_i = k$ is defined as
\begin{align}
	p_k     & \sim k^{- \alpha}
	\ln p_k & \sim - \alpha \ln k + C.
\end{align}
The distribution allows for high number of low-degree nodes and very low number of high-degree nodes.
This is the case of the Internet with a low number of hubs and high amount of users connecting to the hubs.

\subsection{Moments}
The expected value can be calculated as
\begin{align}
	\mathbb{E}\left[d_i\right] & = \sum_{k = 0}^{\infty} k p_k = \sum_k k k^{-\alpha} = \sum_k k^{1-\alpha}.
\end{align}
The sum converges for $\alpha > 2$.

\section{Explain the effect of top heavy distribution of vertex degrees for the networks.}
A minor proportion of highly connected vertices accounts for a sizeable portion of the graph edges.
High clustering coefficient. \\
Such networks are remarkably robust to random node failure, but extremely susceptible to targeted attacks. \\

A fraction of edges attributed to the fraction $p$ of highest degree vertices can be derived from the power law distribution
\begin{align}
	W & = p^{\frac{\alpha - 2}{ \alpha - 1}}
\end{align}
"If I sort the vertices by degree, take only the top 1\% ($\frac{p}{100} \%$) , how many edges will I have."

\section{Explain the importance of the small world effect for the functionality of computer networks.}
\begin{definition}{Small world networks}{}
	The length of the shortest path between any 2 vertices grows with the logarithm of $n$.
\end{definition}
Appears in top-heavy distribution network.
It is very important for the functionality of the Internet - in relatively reasonable amount of hops (10-15), you can reach any computer in the world.
\end{document}